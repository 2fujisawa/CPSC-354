\documentclass{article}

\usepackage{tikz} 
\usetikzlibrary{automata, positioning, arrows} 

\usepackage{amsthm}
\usepackage{amsfonts}
\usepackage{amsmath}
\usepackage{amssymb}
\usepackage{fullpage}
\usepackage{color}
\usepackage{parskip}
\usepackage{hyperref}
  \hypersetup{
    colorlinks = true,
    urlcolor = blue,       % color of external links using \href
    linkcolor= blue,       % color of internal links 
    citecolor= blue,       % color of links to bibliography
    filecolor= blue,        % color of file links
    }
    
\usepackage{listings}
\usepackage[utf8]{inputenc}                                                    
\usepackage[T1]{fontenc}                                                       

\definecolor{dkgreen}{rgb}{0,0.6,0}
\definecolor{gray}{rgb}{0.5,0.5,0.5}
\definecolor{mauve}{rgb}{0.58,0,0.82}

\lstset{frame=tb,
  language=haskell,
  aboveskip=3mm,
  belowskip=3mm,
  showstringspaces=false,
  columns=flexible,
  basicstyle={\small\ttfamily},
  numbers=none,
  numberstyle=\tiny\color{gray},
  keywordstyle=\color{blue},
  commentstyle=\color{dkgreen},
  stringstyle=\color{mauve},
  breaklines=true,
  breakatwhitespace=true,
  tabsize=3
}

\newtheoremstyle{theorem}
  {\topsep}   % ABOVESPACE
  {\topsep}   % BELOWSPACE
  {\itshape\/}  % BODYFONT
  {0pt}       % INDENT (empty value is the same as 0pt)
  {\bfseries} % HEADFONT
  {.}         % HEADPUNCT
  {5pt plus 1pt minus 1pt} % HEADSPACE
  {}          % CUSTOM-HEAD-SPEC
\theoremstyle{theorem} 
   \newtheorem{theorem}{Theorem}[section]
   \newtheorem{corollary}[theorem]{Corollary}
   \newtheorem{lemma}[theorem]{Lemma}
   \newtheorem{proposition}[theorem]{Proposition}
\theoremstyle{definition}
   \newtheorem{definition}[theorem]{Definition}
   \newtheorem{example}[theorem]{Example}
\theoremstyle{remark}    
  \newtheorem{remark}[theorem]{Remark}

\title{CPSC-354 Report}
\author{Linus Fujisawa \\ Chapman University}

\date{\today} 

\begin{document}

\maketitle

\begin{abstract}
This document collects my notes, homework, and reports for CPSC-354 during the semester.
\end{abstract}

\setcounter{tocdepth}{3}
\tableofcontents

\section{Introduction}\label{intro}
This report will grow during the semester. For Week~1 I looked at the MU Puzzle to practice writing about a simple formal system: an axiom, rules of inference, derivations, and the idea of working \emph{inside} versus \emph{outside} the system. This part is based on Chapter~1 of Hofstadter \cite{heb}.

\section{Week by Week}\label{homework}

\subsection{Week 1}

\subsubsection{Notes}
\begin{itemize}
  \item A \emph{formal system} here consists of strings over $\{M,I,U\}$, an axiom $MI$, and four rules. A \emph{theorem} is any string reachable from the axiom by finitely many rule applications.
  \item Working \emph{inside} the system = generate strings by the rules; working \emph{outside} the system = reason about all possible derivations (e.g., invariants).
\end{itemize}

\subsubsection{Homework}
The MU Puzzle is explained in Chapter~1 \cite{heb}. Here are the rules in my own words:

\paragraph{Rules.}
\begin{align*}
\text{(R1)}\;& xI \rightarrow xIU && \text{(append $U$ if the string ends in $I$)}\\
\text{(R2)}\;& Mx \rightarrow Mxx && \text{(duplicate the part after $M$)}\\
\text{(R3)}\;& \text{replace }III\text{ by }U && \text{(wherever it occurs)}\\
\text{(R4)}\;& \text{delete }UU && \text{(wherever it occurs).}
\end{align*}

\paragraph{Claim.} The string $\mathbf{MU}$ cannot be made from $MI$ by these rules.

\paragraph{Reasoning about the number of I's.}
We can track just how many I’s there are in a string:

\begin{itemize}
  \item Rule 1 adds a U, so the I’s stay the same.
  \item Rule 2 doubles the part after M, so the number of I’s doubles.
  \item Rule 3 removes three I’s at once.
  \item Rule 4 only touches U’s, so the I’s stay the same.
\end{itemize}

We start with $MI$, which has 1 I. Doubling moves us between 1 and 2, taking away 3 doesn’t change that cycle, and the other rules don’t affect the I’s. So the number of I’s will always be either 1 or 2. It will never become 0. Since $MU$ has 0 I’s, it’s impossible to reach it from $MI$.

\paragraph{Tiny example trail.}
\[
MI \xrightarrow{\text{R1}} MIU \xrightarrow{\text{R2}} MIUIU
\]
Here the number of I’s goes $1 \to 1 \to 2$ and never becomes a multiple of $3$.

\section{Essay}


\section{Evidence of Participation}
Joined Discord and placed a question; created GitHub repo with \texttt{report.tex} and compiled \texttt{report.pdf}.

\section{Conclusions}\label{conclusion}

\begin{thebibliography}{99}
\bibitem[HEB]{heb} Douglas R. Hofstadter, \emph{Gödel, Escher, Bach: An Eternal Golden Braid}, Basic Books, 1979. Chapma University
\end{thebibliography}

\end{document}
