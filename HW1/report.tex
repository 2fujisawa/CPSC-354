\documentclass{article}

\usepackage{tikz} 
\usetikzlibrary{automata, positioning, arrows} 

\usepackage{amsthm}
\usepackage{amsfonts}
\usepackage{amsmath}
\usepackage{amssymb}
\usepackage{fullpage}
\usepackage{color}
\usepackage{parskip}
\usepackage{hyperref}
  \hypersetup{
    colorlinks = true,
    urlcolor = blue,       % color of external links using \href
    linkcolor= blue,       % color of internal links 
    citecolor= blue,       % color of links to bibliography
    filecolor= blue,        % color of file links
    }
    
\usepackage{listings}
\usepackage[utf8]{inputenc}                                                    
\usepackage[T1]{fontenc}                                                       

\definecolor{dkgreen}{rgb}{0,0.6,0}
\definecolor{gray}{rgb}{0.5,0.5,0.5}
\definecolor{mauve}{rgb}{0.58,0,0.82}

\lstset{frame=tb,
  language=haskell,
  aboveskip=3mm,
  belowskip=3mm,
  showstringspaces=false,
  columns=flexible,
  basicstyle={\small\ttfamily},
  numbers=none,
  numberstyle=\tiny\color{gray},
  keywordstyle=\color{blue},
  commentstyle=\color{dkgreen},
  stringstyle=\color{mauve},
  breaklines=true,
  breakatwhitespace=true,
  tabsize=3
}

\newtheoremstyle{theorem}
  {\topsep}   % ABOVESPACE
  {\topsep}   % BELOWSPACE
  {\itshape\/}  % BODYFONT
  {0pt}       % INDENT (empty value is the same as 0pt)
  {\bfseries} % HEADFONT
  {.}         % HEADPUNCT
  {5pt plus 1pt minus 1pt} % HEADSPACE
  {}          % CUSTOM-HEAD-SPEC
\theoremstyle{theorem} 
   \newtheorem{theorem}{Theorem}[section]
   \newtheorem{corollary}[theorem]{Corollary}
   \newtheorem{lemma}[theorem]{Lemma}
   \newtheorem{proposition}[theorem]{Proposition}
\theoremstyle{definition}
   \newtheorem{definition}[theorem]{Definition}
   \newtheorem{example}[theorem]{Example}
\theoremstyle{remark}    
  \newtheorem{remark}[theorem]{Remark}

\title{CPSC-354 Report}
\author{Linus Fujisawa \\ Chapman University}

\date{\today} 

\begin{document}

\maketitle



\setcounter{tocdepth}{3}

\section*{The MU Puzzle}

The MU Puzzle, introduced in Chapter 1, starts with a very simple setup the axiom string \textbf{MI} and four rules that let us change or extend the string using the letters M, I, and U. The puzzle asks whether it’s possible to reach the string \textbf{MU} by applying these rules.

At first glance it seems like it might work if you just keep trying different combinations, but after some trial and error this problem will not have a solution. The key observation is that no matter what rule you apply, the number of I’s never becomes a multiple of three. It always stays in a certain pattern.

\begin{itemize}
  \item The starting axiom $MI$ has one I.
  \item Rule 1 ($xI \to xIU$) adds a U at the end, but doesn’t change the number of I’s.
  \item Rule 2 ($Mx \to Mxx$) doubles the part after the M, which changes the number of I’s but never makes it a multiple of three.
  \item Rule 3 ($III \to U$) removes three I’s, so the overall pattern stays the same.
  \item Rule 4 ($UU \to$ delete) only affects U’s, not I’s.
\end{itemize}


So the system never produces a string where the number of I’s is a multiple of three. Since $MU$ has zero I’s, it doesn’t fit the pattern, which means there’s no way to reach it from $MI$.

\par
For example, starting with $MI$ (which has 1 I), Rule 1 gives $MIU$. This still has 1 I. If we keep going, $MIU \to MIUIU$ using Rule 2, the number of I’s doubles to 2. At no point does the number of I’s ever become a multiple of three, which is why $MU$ can’t be reached.

\section*{Conclusion}
In the end, the MU puzzle has no solution. The rules simply don’t allow us to create $MU$ starting from $MI$. By spotting the pattern with the I’s, we can prove once and for all that $MU$ is unreachable.




\end{document}
